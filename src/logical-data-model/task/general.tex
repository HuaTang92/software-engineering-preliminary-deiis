
\section{Knowing basic concepts}

From
Wikipedia\footnote{\url{http://en.wikipedia.org/wiki/Named-entity_recognition}},
the process of named entity recognition is defined as:

\begin{quote}
Named-entity recognition (NER) is a subtask of information extraction that seeks
to locate and classify atomic elements in text into predefined categories such
as the names of persons, organizations, locations, expressions of times,
quantities, monetary values, percentages, etc.
\end{quote}

Several things you should be aware:

\begin{enumerate}

\item NER involves two subtasks: ``locating'' and ``classifying'', which
correspond to the identification of the text span (i.e. begin and end) and the
assignment of a label to the text respectively. We can see NER task perfectly
fits into the UIMA framework.

\item Named entities correspond to the atomic elements in text, which are not
necessarily single tokens (those you finally print out to the console from
Homework 0). Many times, named entities can correspond to several consecutive
tokens, and many tokens may not be part of named entities.

\item NER plays a very crucial role in not only information extraction, but in
all the text analysis related task, e.g. information retrieval and question
answering. In the next homework, you will find NER can help you better identify
the key words in the query and documents, retrieve synonyms more accurately, and
thus relevant document more effectively.
 
\end{enumerate}
