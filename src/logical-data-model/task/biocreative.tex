
\section{Learning gene mention tagging}

For this task, you are going to tackle a specific NER problem in the biological
domain, \emph{Gene Mention
Tagging}\footnote{\url{http://www.biocreative.org/tasks/biocreative-ii/task-1a-gene-mention-tagging/}},
and you are going to use partial labeled dataset from BioCreAtIvE (Critical
Assessment of Information Extraction systems in Biology challenge
evaluation)\footnote{\url{http://www.biocreative.org}}, which focuses on
evaluating text mining and information extraction systems applied to the
biological domain. Gene Mention Tagging task is concerned with the named entity
extraction of gene and gene product mentions in text. For example, given a
sentence from a biological literature: \emph{Comparison with alkaline
phosphatases and 5-nucleotidase}, you should identify \emph{alkaline
phosphatases} and \emph{5-nucleotidase} correspond to gene names.

The input file will consist of ascii sentences, one per line. Each sentence will
be preceded on the same line by a sentence identifier, i.e.,

\begin{verbatim}
sentence-identifier-1 text
\end{verbatim}

For example, the sentence in the above example will be given to you as follows,

\begin{verbatim}
P00001606T0076 Comparison with alkaline phosphatases and 5-nucleotidase
\end{verbatim}

You are given an example input file (consisting of 15,000 sentences, each per
line, all from biological literature). We will later test your pipeline with a
different dataset.

You are required to output an ascii list of reported gene name mentions, one per
line, and formatted as:

\begin{verbatim}
sentence-identifier-1|start-offset-1 end-offset-1|optional text... 
sentence-identifier-1|start-offset-2 end-offset-2|optional text... 
sentence-identifier-1|start-offset-3 end-offset-3|optional text... 
sentence-identifier-2|start-offset-1 end-offset-1|optional text... 
sentence-identifier-3|start-offset-1 end-offset-1|optional text... 
. 
. 
. 
\end{verbatim}

For example, we expect you to generate the following to lines for the example
sentence above:

\begin{verbatim}
P00001606T0076|14 33|alkaline phosphatases
P00001606T0076|37 50|5-nucleotidase
\end{verbatim}

The sentence-identifier is from the sentence of the mention. Multiple mentions
from the same sentence should appear on separate lines. A sentence is not
required to have any mentions. The start-offset is the number of non-whitespace
characters in the sentence preceding the first character of the mention, and the
end-offset is the number of non-whitespace characters in the sentence preceding
the last character of the mention. A sample output file will also be given to
you, which contains all the identified gene mentions (i.e. gold-standard) for
the input file.
