
\section{Adding a type system descriptor and Java classes for your pipeline}

If you have one or several idea(s) for the sample information processing task
already, it's time to implement all of them in this task. You will find the
official tutorial is helpful for you to accomplish this task. Except that you
are required to build your component based on UIMA framework, there is little
limitation on how you should design and implement your type system.
Nevertheless, there are still some suggestions (or you can say tips or our
expectations) to consider.

\begin{enumerate}

\item The minimal type system to accomplish this task should include types for
input and output elements, i.e. the id and text of each sentence and gene tags.

\item If you intend to design a pipeline of two or more phases (e.g., each of
the first few phases extracts a certain feature, and the last phase predicts the
gene mentions based on all the features), you should include all the
intermediate types in your type system as well.

\item If your type system is complex, then you can try to refactor the type
system by categorizing the types according to their functions (e.g., NLP
related, biological related, etc.), and creating multiple type systems for each
of the categories.

\item Another common practise in the type system design is to create a
\emph{base annotation} type which includes two features: a string feature
\emph{source} and a numeric feature \emph{confidence} to help keep track of
where an annotation was originally made by, and how confidence the annotation
was. All other types then inherit from this base annotation type.

\item You could assign a (or several) name space(s) for your types, e.g.,
instead of \texttt{Sentence} (default name space), you could name it
\texttt{model.Sentence} (a type with name space \texttt{model}). Once you click
the \textbf{JCasGen} button, you will find a nice hierarchical folder structure
created under \texttt{src/main/java} for the Java classes corresponding to the
types. We also recommend to follow the same naming convention for your artifact
groups, Java packages, and type paths.

\end{enumerate}