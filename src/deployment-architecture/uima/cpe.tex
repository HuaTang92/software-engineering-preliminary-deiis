
\chapter{Execution Architecture with CPE}


In this task, you will need to learn what a Collection Processing Engine (CPE) is and how to use it.
You are required to run your pipeline with a CPE instead of the UIMA Document Analyzer. 

\section{Learning CPE}

There are several useful resources to help you get started with CPE:
\begin{itemize}
  \item You can learn basic concepts and usage about CPE from 
				\emph{Chapter 2. Collection Processing Engine Developer's Guide} 
				(\url{http://uima.apache.org/d/uimaj-2.4.0/tutorials_and_users_guides.html#ugr.tug.cpe}). 
	   \item Also, you can view the manual for CPE GUI 
	   (\url{http://uima.apache.org/d/uimaj-2.4.0/tools.html#ugr.tools.cpe}).
\end{itemize}

\section{Creating and Running your CPE}

\begin{enumerate}

\item Ideally, depending on where the input file is located (on a local file
system, in a jar file, or from an external sources, e.g. network), your
collection reader needs to be general enough to establish a connection to the
file source and open a stream to read the content. But for our task, you only
need to consider a folder of files located on the file system, which means 
you can use org.apache.uima.tools.components.FileSystemCollectionReader directly, 
and only need to write a Collection Reader descriptor to fit your needs.  

\item You are required to create a Cas Consumer based on the Evaluator component of homework 2, 
and include it in your CPE pipeline.

\item Please name your CPE descriptor as hw3-ID-CPE.xml and put it
under \texttt{src/main/resources/}, so that we could easily find the entry point
of your pipeline.

 
\end{enumerate}