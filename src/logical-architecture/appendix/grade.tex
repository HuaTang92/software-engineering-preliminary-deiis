
\subsection{Grading Guidelines}

\subsubsection{Designing and Implementing UIMA Analysis Engine (60 pts)}

\begin{itemize}

\item We will evaluate your system for the necessary IIS steps and its overall structure whether designed and implemented in a proper way, by looking at your
codes (Java classes and analysis descriptors) and report. 
Specifically, each of following items carry 50\% design and 50\% implementation points: 
Overall Structure, Test Element Annotation, Token Annotation, NGram Annotation, Answer Scoring, Evaluation and any other annotators you might have added.

\item If the basic requirement (i.e., it works!) for each of the items is met,
then the student can get upto 21-30 points, otherwise 0-20 points will be given.

\item If the design and implementation of each annotator is considered more carefully (according to the suggestions given in the homework instruction or outside), then upto 31-50 pts points will be given. For example, the type system is implemented in an inherited way.

\item Out of box thinking can earn you up to 60 pts in this part of the homework.

\end{itemize}

\subsubsection{Documentations, comments, coding style (30 pts)}

\begin{itemize}

\item We will first look into your report to see if it is complete, if you have
given explanations (examples) and solutions to all our concerns, and then we will take a look at the Javadoc package for your project to see if it covers all the necessary information that potential users of your package need to know.

\item Basic points are up to 15-20 for report, if some
important part(s) is missing, the 0-15 points might be given.

\item A detailed documentation can be given up to 30 pts, for example,

\begin{enumerate}
\item good insights to the problem or architecture design covered in the report,
\item apply design pattern in an appropriate way,
\item detailed comparison of more than one methods in the report,
\item interesting discovery,
\item etc.
\end{enumerate}

\end{itemize}

\subsubsection{Submission, folder structure, and name convention (10 pts)}

\begin{itemize}

\item We will first check the correctness of your submission (5 pts), 
and we will look into your report, code and descriptor to see whether they follow 
the normal Java name convention 
%\footnote{\url{http://www.oracle.com/technetwork/java/javase/documentation/codeconvtoc-136057.html}} (5 pts).

\item Specifically, we expect your homework is located at the right repository 
and your project folders/files are organized as required (5 pts). 
0 to 4 points might be given if you submitted to wrong path and/or badly organized folder/files. 

\item The names of your types and files should follow the normal Java name convention (5 pts). 
Otherwise, 0 to 4 points will be given for bad naming. 


\end{itemize}
