
\section{Adding analysis engines for your pipeline}

You are required to use the type system we included in the archetype to
accomplish this homework. However, if your implementation needs additional 
types (e.g., to store intermediate data), 
you are welcome to extend the type system (don't forget to regenerate Java classes in this case
you need to use \texttt{JCasGen}).

The ultimate goal is (1) implement an aggregated analysis engine (hw2-ID-aae.xml) with several analysis engines
that annotate the inputs;
(2) evaluate the performance of the aggregate analysis engine by
comparing an output of your system against the gold standard.
Remember that scoring in this assignment puts more weight on performance!

We will evaluate your implementation using held-out data.
\begin{itemize}
\item The input file should be \texttt{src/main/resources/data/hw2.in} ;
\item The output file should  \texttt{src/main/resources/data/hw2-ID.out},
where \texttt{ID} is \textbf{your Andrew ID};
\item The aggregate analysis engine should be defined by 
the descriptor \texttt{hw2-ID-aae.xml}.
Please, place it in the folder   \texttt{src/main/resources/}.
\end{itemize}

You can test your aggregated analysis engine by running 
\href{http://uima.apache.org/d/uimaj-2.4.0/tools.html#ugr.tools.doc_analyzer}{the UIMA Document Analyzer}.
One option is to use the training data from Homework 1, but you are also encouraged
to obtain other test sets.

We provide an Eclipse configuration to run the document analyzer,
which is \newline\texttt{run\_configuration/UIMA Document Analyzer.launch}. 
To use this configuration in Eclipse, navigate to the project sub-folder \texttt{run\_configuration},
and right click on \texttt{UIMA Document Analyzer.launch}.
Then, select \texttt{Run As}.


~\\
Here are some suggestions to consider.

\begin{enumerate}

\item Remember to add comments and Javadocs to your annotators, we will also
evaluate the quality of your code.

\item If you want to employ a resource (e.g., a model you trained offline) in
your annotator, you could consider UIMA's \emph{resource manager} (refer to the
official tutorial for details about this).
Be sure to put your resource in \texttt{src/main/resources} so 
that your this resource will be properly bundled up with your code
during the submission/release process.

\item If you want to incorporate other NLP or machine learning tools,
\textbf{do not use non-Java} packages, as we will not be able to use them
on an evaluating server. 
However, we might be able to post jar packages 
to our 3rd party repository (and make them available to you via maven).

\item The most creative part is to implement specific annotators. We
outline some possible solutions (in Homework 1). 
Note that an aggregate analysis engine may contain only a single
primitive analysis engine (and a single annotator). 
Yet, it is \textbf{best to implement multiple solutions}.

\item All the annotations should be kept in the CAS until a final
``merging'' component reads all the annotations and selects only
the best ones (or all of them as one option).
You should mark gene mentions using the type \emph{edu.cmu.deiis.types.Annotation}.
The \emph{casProcessorId} feature 
can be used to indicate a type of the annotator that have tagged a gene name, 
and the \emph{confidence} feature 
can be used to indicate 
an estimated annotation quality (the higher is the confidence value the better
is the annotation).

Note that many statistical NERs produce such confidence values.
For rule-based annotators, you can use some ad hoc fixed value, e.g., one.
You can use these confidence values to aggregate results from several annotators.
For example, you can employ a voting procedure 
in which confidence values are used as voting weights.
If a single annotation is produced by multiple annotators it will have a larger weight
(a sum of individual weights from all contributing annotators)
and a better chance to be selected by the aggregating algorithm (which, e.g.,
can use a threshold to reject low-quality annotations).
This approach was used in IBM Watson system.

\end{enumerate}
