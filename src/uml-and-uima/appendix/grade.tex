
\newpage

\subsection{Grading Guidelines}

\subsubsection{Task 1: UML documents (20 pts)}

\begin{itemize}

\item Each problem is worth 10 pts. \\
 
\end{itemize}

\subsubsection{Task 3: Design \& Implementation of IIS (80 pts)}

\textbf{IIS design \& architecture (20 pts)}
 
\begin{itemize}

\item We will evaluate if the necessary IIS components: collection reader (5
pts), annotator (5 pts), cas consumer (5 pts), and the CPE descriptor and type
system (5 pts) are designed and implemented in a proper way, by looking at your
codes (Java classes and descriptors) and report.

\item If the basic requirement (i.e., it works!) for each of the items is met,
then the student can get 3 points, otherwise 0 to 2 points will be given.

\item If the design of each component is considered more carefully (according to
the suggestions given in the homework instruction or outside), then 4 or 5 pts
points will be given. For example,

\begin{enumerate}
\item line-by-line collection reader,
\item id and text are separated during reading collection,
\item loading a pretrained model in the initialize phase instead of process phrase,
\item whitespace-excluded offsets are calculated only until cas consumer,
\item proper use of configuration parameters instead of hard-wired model/file path,
\item proper use of multi-view,
\item type system is implemented in an inherited way,
\item etc.
\end{enumerate}

\end{itemize}

\textbf{Documentations, comments, coding style (30 pts)}

\begin{itemize}

\item We will first look into your report to see if it is complete, if you have
given explanations and solutions to all our concerns (20 pts), and then we will
take a look at the Javadoc package the Maven automatically generated for your
project to see if it covers all the necessary information that potential users
of your package need to know (5 pts), and finally we will again look into your
source package to see if your Java classes are coded according to Java's best
practise (5 pts).

\item Basic points for each category is 3 pts (6 pts for the report), if some
important part is missing, the 0 to 2 points (or 0 to 5 points for the report)
might be given.

\item A detailed document can be given 4 or 5 points (7 to 10 points for report), for example,

\begin{enumerate}
\item good insights to the problem or architecture design covered in the report,
\item include UML design or other forms of software design in the report or Javadoc,
\item apply design pattern in an appropriate way,
\item detailed comparison of more than one methods in the report,
\item interesting discovery from the experiments,
\item etc.
\end{enumerate}

\end{itemize}

\textbf{Algorithm design \& performance (30 pts)}

\begin{itemize}

\item We will first evaluate your submission to see if can produce some
results (20 pts), and we will look into your report and code to find out what
algorithms or knowledge base you are using (10 pts).

\item For each 0.1 in performance, you will be given 1 point. We don't expect performance
to be much higher than 0.7

\item Additional points (3 points) will be given for the algorithm design.
For example, if the students consider the following aspects in their algorithm
design,

\begin{enumerate}
\item wrap more than one methods for NER and pick the best one,
\item a method (maybe another annotator down the pipeline) to combine different methods
\item proper use of confidence (probability) of knowledge sources or tools,
\item use existing tool but train a model based on a different corpus,
\item implement evaluation components,
\item propose and implement novel ideas for the task,
\item investigate the problem from a biomedical (instead of language technology) view,
\item etc.
\end{enumerate}


\end{itemize}
